\documentclass{article}
\usepackage{jdaghw}

\newcommand{\hmwkTitle}{Example problem set}
\newcommand{\hmwkDueDate}{1970-01-01}
\newcommand{\hmwkClass}{BMES ???}
\newcommand{\hmwkClassInstructor}{A. Teacher}
\newcommand{\hmwkAuthorName}{J.D.A. Gilliland}
\newcommand{\hmwkStudentID}{13462252}
\newcommand{\hmwkClassTime}{Sometime} % Class/lecture time

\begin{document}
\maketitle

\begin{multicols}{2}
\section*{Problem 1}
This is my work on problem 1.

\section*{A sample code listing}
For the code used to generate said plot, see
listing \ref{lst:exer1}.

\end{multicols}

\begin{figure}
  \centering
  \includegraphics[width=0.8\textwidth]{build/cobweb.png}
  \caption{
    Cobweb plots for problem 2.
    The green line is the finite difference function $f(x_n)$, and the blue
    line is the line $x=y$, or $x_n = x_{n+1}$.
    For part a, we assume $r=0.5$.
  }
  \label{fig:cobweb}
\end{figure}

\begin{listing}[ht]
   \begin{framed}
    \input{build/code_sample.tex}
    \caption{The code used.}
    \label{lst:exer1}
   \end{framed}
\end{listing}


\begin{table}
  \csvloop{
  file=src/data_numbers.csv,
  head to column names,
  before reading=\centering\sisetup{table-number-alignment=center},
  tabular={lSS[table-format=2.2e1]},
  table head=\toprule\textbf{Month} & \textbf{Dogs} & \textbf{Cats}\\\midrule,
  command=\month & \dogs & \cats,
  table foot=\bottomrule}
  \caption{csvloop table.}
  \label{tab:data-numbers-loop}
\end{table}

\end{document}
